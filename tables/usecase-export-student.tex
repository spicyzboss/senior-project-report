\begin{table}[H]
    \caption{Use Case ส่งออกรายชื่อนักเรียน}
    \label{tab:usecase-export-student}
    \begin{tabularx}{\textwidth}{ | p{3cm} | X | }
    \hline
    \textbf{Use Case} & ส่งออกรายชื่อนักเรียน \\
    \hline
    \textbf{Actor} & ผู้ประสานงาน, ผู้ดูแลระบบ \\
    \hline
    \textbf{Description} & ผู้ใช้สามารถส่งออกรายชื่อนักเรียนได้ \\
    \hline
    \textbf{Pre-Condition} & - \\
    \hline
    \textbf{Post-Condition} & - \\
    \hline
    \end{tabularx}
    \begin{tabularx}{\textwidth}{ | X | X | X | }
    \multicolumn{3}{|c|}{\textbf{Flow of Events}} \\
    \hline
    \multicolumn{1}{|c|}{\textbf{General}} & \multicolumn{1}{|c|}{\textbf{Actor Action}} & \multicolumn{1}{|c|}{\textbf{System Response}} \\
    \hline
    1. ผู้ใช้คลิกที่การ์ดนักเรียนเพื่อเข้าสู่ Use Case นี้ &  &  \\
    \hline
    & 2. ผู้ใช้คลิกปุ่มดาวน์โหลดรายชื่อนักเรียน  &  \\
    \hline
    & 3. ผู้ใช้คลิกเลือกรูปแบบไฟล์ที่ต้องการ  &  \\
    \hline
    & & 4. ระบบเลือกรายชื่อนักเรียนของโรงเรียนผู้ประสานงานจากฐานข้อมูลสัมพันธ์ \\
    \hline
    & & 5. ระบบแปลงไฟล์เป็นรูปแบบเดียวกับที่ผู้ใช้เลือก \\
    \hline
    & & 6. ระบบดาวน์โหลดรายชื่อให้ผู้ใช้ \\
    \hline
    \end{tabularx}
\end{table}
