\begin{table}[H]
    \caption{Use Case แก้ไขข้อมูลนักเรียน}
    \label{tab:usecase-edit-student}
    \begin{tabularx}{\textwidth}{ | p{3cm} | X | }
    \hline
    \textbf{Use Case} & แก้ไขข้อมูลนักเรียน \\
    \hline
    \textbf{Actor} & ผู้ประสานงาน \\
    \hline
    \textbf{Description} & ผู้ใช้สามารถแก้ไขข้อมูลนักเรียนได้ \\
    \hline
    \textbf{Pre-Condition} & - \\
    \hline
    \textbf{Post-Condition} & - \\
    \hline
    \end{tabularx}
    \begin{tabularx}{\textwidth}{ | X | X | X | }
    \multicolumn{3}{|c|}{\textbf{Flow of Events}} \\
    \hline
    \multicolumn{1}{|c|}{\textbf{General}} & \multicolumn{1}{|c|}{\textbf{Actor Action}} & \multicolumn{1}{|c|}{\textbf{System Response}} \\
    \hline
    1. ผู้ใช้คลิกที่การ์ดนักเรียนเพื่อเข้าสู่ Use Case นี้ & & \\
    \hline
    & 2. ผู้ใช้คลิกที่ปุ่มแก้ไขของนักเรียนที่ต้องการ & \\
    \hline
    & 3. ผู้ใช้กรอกข้อมูลของที่ต้องการแก้ไขลงในกล่องข้อความแต่ละกล่อง & \\
    \hline
    & 4. ผู้ใช้คลิกที่ปุ่มแก้ไข & \\
    \hline
    & & 5. ระบบแก้ไขข้อมูลนักเรียนลงในฐานข้อมูลสัมพันธ์ \\
    \hline
    & & 6. ระบบแจ้งเตือนการแก้ไขข้อมูลนักเรียนสำเร็จให้แก่ผู้ใช้ \\
    \hline
    \end{tabularx}
\end{table}
