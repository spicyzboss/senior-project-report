\begin{table}[H]
    \caption{Use Case อ่านประเภทการแข่งขัน}
    \label{tab:usecase-read-contest}
    \begin{tabularx}{\textwidth}{ | p{3cm} | X | }
    \hline
    \textbf{Use Case} & อ่านประเภทการแข่งขัน \\
    \hline
    \textbf{Actor} & ผู้ประสานงาน, ผู้ดูแลระบบ \\
    \hline
    \textbf{Description} & ผู้ใช้สามารถอ่านประเภทการแข่งขันได้ \\
    \hline
    \textbf{Pre-Condition} & - \\
    \hline
    \textbf{Post-Condition} & - \\
    \hline
    \end{tabularx}
    \begin{tabularx}{\textwidth}{ | X | X | X | }
    \multicolumn{3}{|c|}{\textbf{Flow of Events}} \\
    \hline
    \multicolumn{1}{|c|}{\textbf{General}} & \multicolumn{1}{|c|}{\textbf{Actor Action}} & \multicolumn{1}{|c|}{\textbf{System Response}} \\
    \hline
    1. ผู้ใช้คลิกที่การ์ดประเภทการแข่งขันเพื่อเข้าสู่ Use Case นี้ &  &  \\
    \hline
    & & 2. ระบบสร้างการร้องขอข้อมูลประเภทการแข่งขันจากฐานข้อมูลสัมพันธ์และสร้างเป็นตารางประเภทการแข่งขันให้ผู้ใช้ \\
    \hline
    \end{tabularx}
\end{table}
