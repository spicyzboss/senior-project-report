\begin{table}[H]
    \caption{Use Case สร้างประเภทการแข่งขัน}
    \label{tab:usecase-create-contest}
    \begin{tabularx}{\textwidth}{ | p{3cm} | X | }
    \hline
    \textbf{Use Case} & สร้างประเภทการแข่งขัน \\
    \hline
    \textbf{Actor} & ผู้ดูแลระบบ \\
    \hline
    \textbf{Description} & ผู้ดูแลระบบสามารถสร้างประเภทการแข่งขันได้ \\
    \hline
    \textbf{Pre-Condition} & - \\
    \hline
    \textbf{Post-Condition} & - \\
    \hline
    \end{tabularx}
    \begin{tabularx}{\textwidth}{ | X | X | X | }
    \multicolumn{3}{|c|}{\textbf{Flow of Events}} \\
    \hline
    \multicolumn{1}{|c|}{\textbf{General}} & \multicolumn{1}{|c|}{\textbf{Actor Action}} & \multicolumn{1}{|c|}{\textbf{System Response}} \\
    \hline
    1. ผู้ดูแลระบบคลิกที่การ์ดประเภทการแข่งขันเพื่อเข้าสู่ Use Case นี้ &  &  \\
    \hline
    & 2. ผู้ดูแลระบบคลิกปุ่มสร้างประเภทการแข่งขัน  &  \\
    \hline
    & 3. ผู้ดูแลระบบกรอกข้อมูลประเภทการแข่งขัน  &  \\
    \hline
    & 4. ผู้ดูแลระบบกดปุ่มสร้าง &  \\
    \hline
    & & 5. ระบบเพิ่มข้อมูลประเภทการแข่งขันลงในฐานข้อมูลสัมพันธ์ \\
    \hline
    & & 6. ระบบแจ้งเตือนการเพิ่มประเภทการแข่งขันสำเร็จแก่ผู้ดูแลระบบ \\
    \hline
    \end{tabularx}
\end{table}
