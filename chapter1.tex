\chapter{บทนำ}
\label{chapter:introduction}

\section{ที่มาและความสำคัญ}

ในปัจจุบันคณะเทคโนโลยีสารสนเทศ สถาบันเทคโนโลยีพระจอมเกล้าเจ้าคุณทหารลาดกระบัง ได้สิทธิ์เป็นผู้จัดโครงการแข่งขันทดสอบทักษะการคิดเชิงคำนวณ (Bebras Computational Thinking Challenge) และให้บริการเว็บไซต์และระบบสำหรับสมัครเข้าแข่งขัน ผู้จัดทำได้เล็งเห็นถึงปัญหาและข้อบกพร่องของระบบแสดงสถิติและจัดการรายชื่อของผู้สมัครแข่งขันที่มีอยู่ในปัจจุบัน มีฟังก์ชันการทำงานที่ไม่สมบูรณ์ทั้งยังมีบางฟังก์ชันการทำงานที่ไม่สามารถใช้งานได้จริงและมีการใช้งานบางส่วนที่สามารถนำมาพัฒนาต่อได้ ทั้งนี้ทางคณะผู้จัดทำจึงได้ทำการสรุปความต้องการของผู้ใช้และออกแบบโครงสร้างและระบบของเว็บไซต์ขึ้นมาใหม่ เพื่อให้ตรงตามความต้องการของผู้ใช้ โดยให้สามารถใช้งานได้เสถียรยิ่งขึ้น ตอบสนองต่อผู้ใช้งานได้ดีมากขึ้น และยังสามารถใช้งานได้ง่ายมากขึ้น

\section{วัตถุประสงค์}
\begin{enumerate}
    \item เพื่อศึกษาและค้นคว้าเกี่ยวกับการออกแบบและพัฒนาระบบเพื่อนำมาใช้จริง
    \item เพื่อศึกษาการออกแบบส่วนต่อประสานและพัฒนาประสบการณ์การใช้งานของผู้ใช้งานบนเว็บไซต์ ให้มีความเหมาะสมตามหลักการออกแบบ
    \item เพื่อแก้ไขปัญหาและปรับปรุงระบบแสดงสถิติและจัดการรายชื่อของผู้สมัครแข่งขันทดสอบทักษะการคิดเชิงคำนวณ ให้สามารถใช้งานได้จริงและตอบสนองต่อผู้ใช้งานได้อย่างถูกต้อง
    \item เพื่อสร้างระบบสามารถพัฒนาและปรับปรุงเพิ่มเติมได้ง่าย
\end{enumerate}

\section{ขอบเขตของโครงงาน}
\begin{enumerate}
    \item สามารถนำเข้าและนำออกข้อมูลของผู้สมัครแข่งขันเป็นไฟล์ Excel (.xlsx) ได้
    \item สามารถเพิ่ม-ลบ-แก้ไขรายชื่อผู้สมัครแข่งขันได้
    \item สามารถให้ผู้สมัครแข่งขันทดสอบเลือกประเภทการทดสอบได้
    \item สามารถค้นหาและคัดกรองรายชื่อผู้เข้าสมัครแข่งขันได้
    \item สามารถใช้งานได้ง่ายบนรูปแบบเว็บแอปพลิเคชัน
    \item สามารถนำเข้าและนำออกข้อมูลจาก Moodle ได้
    \item สามารถจัดเก็บและแสดงข้อมูลสถิติผลการแข่งขันในแต่ละครั้งได้
\end{enumerate}

\section{ขั้นตอนการดำเนินงานของโครงงาน}
\begin{enumerate}
    \item รวบรวมและวิเคราะห์ความต้องการของผู้ใช้งาน
    \item ร่วมกันพูดคุยกับทีมเรื่องเทคโนโลยีที่ใช้
    \item ออกแบบฐานข้อมูล
    \item ศึกษาการใช้งานของเทคโนโลยีที่ใช้
\end{enumerate}

\section{ขั้นตอนการดำเนินงานของโครงงาน}
\begin{enumerate}
    \item รวบรวมและวิเคราะห์ความต้องการของผู้ใช้งาน
    \item ร่วมกันพูดคุยกับทีมเรื่องเทคโนโลยีที่ใช้
    \item ออกแบบฐานข้อมูล
    \item ศึกษาการใช้งานของเทคโนโลยีที่ใช้
\end{enumerate}

\section{ประโยชน์ที่คาดว่าจะได้รับ}
\begin{enumerate}
    \item ได้รับประสบการณ์ในการออกแบบ ปรับปรุง และพัฒนาระบบ ที่ทันสมัยและสามารถนำมาใช้งานได้จริง
    \item ได้รับทักษะในค้นคว้าและเก็บข้อมูลจากผู้ใช้งานจริง เพื่อนำมาพัฒนาระบบที่สามารถใช้งานได้จริง
\end{enumerate}

\subsection{การเขียนบรรณานุกรม}

เป็นส่วนที่แสดงถึงการศึกษาค้นคว้าวิจัยของผู้เขียนว่า มีความสมบูรณ์กว้างขวาง ลึกซึ้ง ทันสมัยน่าเชื่อถือมากน้อยเพียงใด โดยทั่วไป คำว่า บรรณานุกรม รวบรวมรายการเอกสารสิ่งพิมพ์ โสตทัศน์ สื่ออิเล็กทรอนิกส์ ที่ผู้เขียนศึกษาค้นคว้าทั้งหมด แม้ว่าจะไม่ได้คัดลอกข้อความมา ส่วนคำว่า เอกสารอ้างอิงนิยมใช้กับรายการเอกสาร สิ่งพิมพ์ โสตทัศน์ เฉพาะที่คัดลอกและยกมาอ้างอิงในเนื้อหา
บรรณานุกรมเป็นส่วนประกอบสำคัญส่วนหนึ่งของการผลิตสิ่งพิมพ์โดยเฉพาะสิ่งพิมพ์ประเภทตำราหรือหนังสือเรียน และเอกสารประกอบการสอน เพราะเป็นส่วนที่ระบุถึงหนังสือและเอกสารต่าง ๆ ตลอดจนการสัมภาษณ์ที่ผู้เขียนใช้เป็นแหล่งข้อมูลในการอ้างอิง

\textbf{ตัวอย่างการแทรกบรรณานุกรมที่อ้างอิงถึง}

\cite{xieRethinkingSpatiotemporalFeature2018}
\cite{Goodfellow-et-al-2016}
\cite{sunHumanActionRecognition2015}
\cite{wangDeepFaceRecognition2018}
\cite{burtWebVideoPhotos2019}
\cite{sagonasSemiautomaticMethodologyFacial2013}
\cite{fungEndToEndLowResourceLipReading2018}
\cite{petridisVisualOnlyRecognitionNormal2018}
\cite{tranCloserLookSpatiotemporal2017}
\cite{xiaogangwangBoostedMultitaskLearning2009}
\cite{nvidiaTensorFlowDeterminism}
