\chapter{วิธีการดำเนินการวิจัย}
\label{chapter:experiment}

\section{ความต้องการของระบบ}

หลังจากที่ได้สอบถามผู้ใช้งานและที่ปรึกษาระบบบีบราส พวกเรานั้นได้ความต้องการออกมาเป็นสองระบบ มีรายละเอียดดังนี้

\subsection{ระบบสามารถแสดงสถิติได้}

ระบบจะต้องสามารถแสดงสถิติการแข่งขันทุกครั้งที่ผ่านมาได้
โดยจะต้องสามารถดูสถิติการแข่งขันทดสอบทักษะการคิดเชิงคำนวณได้ทั้งระดับมัธยมต้นและมัธยมปลายของโรงเรียนของผู้ใช้งานได้
และจะต้องสามารถแสดงสถิติการแข่งขันโดยรวมของโรงเรียนทั้งหมดได้

\subsection{ระบบจัดการรายชื่อของผู้สมัคร}

ระบบจัดการรายชื่อของผู้สมัครแบ่งเป็น 3 ระบบย่อย ได้แก่

\begin{enumerate}
    \item ระบบจะต้องสามารถเพิ่มข้อมูลของผู้สมัครลงในฐานข้อมูลได้
    \item ระบบจะต้องสามารถลบข้อมูลของผู้สมัครออกจากฐานข้อมูลได้
    \item ระบบจะต้องสามารถแก้ไขข้อมูลของผู้สมัครได้
\end{enumerate}

\section{คำอธิบายของระบบ}

ระบบแสดงสถิติและจัดการรายชื่อของผู้สมัครแข่งขันทดสอบทักษะการคิดเชิงคำนวณแบ่งออกเป็น 2 ส่วน ได้แก่

\begin{enumerate}
    \item \textbf{ส่วนแสดงสถิติการแข่งขัน} ผู้ใช้สามารถเลือกที่การ์ดการแข่งขัน (Contest) เพื่อให้แสดงข้อมูลการแข่งขันได้โดยเบื้องต้นระบบจะแสดงเป็นข้อมูลสถิติการแข่งขันทั้งหมดและข้อมูลสถิติการแข่งขันของโรงเรียนของผู้ใช้งาน โดยผู้ใช้สามารถเลือกที่จะค้นหา (Search) สถิติการแข่งขันของแต่ละโรงเรียนหรือแต่ละบุคคลได้ เมื่อทำการค้นหา ระบบจะแสดงข้อมูลสถิติตามการค้นหานั้น
    \item \textbf{จัดการรายชื่อของผู้สมัครแข่งขัน} ในกรณีที่ผู้ใช้งานคือผู้ดูแลระบบ (Admin) หรือผู้ใช้งานเป็นผู้ประสานงาน (Coordinator) สามารถเลือกที่การ์ดนักเรียน (Student) เพื่อจัดการกับรายชื่อนักเรียนได้
\end{enumerate}

\newpage

\section{การออกแบบระบบจากความต้องการ}
\subsection{ภาพรวมของระบบ}

ในระบบแสดงสถิติและจัดการรายชื่อของผู้สมัครแข่งขันทดสอบทักษะการคิดเชิงคำนวณได้ใช้โครงสร้างการทำงานแบบไมโครเซอร์วิส (Microservices) \cite{WhatAreMicroservices} เพื่อให้สามารถแยกส่วนการนำส่งระบบ (Deployment) ได้ง่าย โดยมีโครงสร้างการออกแบบดังแสดงในรูปที่ \ref{fig:simplified-system-overview-diagram}

\begin{figure}[H]
    \centering
    \includegraphics[width=125mm,scale=1.0]{diagrams/simplified-system-overview-diagram.png}
    \caption{ภาพรวมของระบบอย่างง่าย}
    \label{fig:simplified-system-overview-diagram}
\end{figure}

\textbf{คำอธิบาย}

สำหรับแต่ละ Use Case เว็บจะสร้างการร้องขอไปที่ API แต่ละตัว เพื่อให้สามารถทำงานได้ตามความต้องการของระบบ และสามารถอ่าน-เขียนข้อมูลลงไปที่ฐานข้อมูลสัมพันธ์ได้

\subsection{ระบบแสดงสถิติ}

ผู้จัดทำได้สกัดความต้องการของระบบแสดงสถิติออกมาเป็น Use Case ทั้งหมด 6 อัน โดยจะแบ่งเป็น 2 บทบาทการใช้งาน คือ ผู้ประสานงานและผู้ดูแลระบบ ดังแสดงในรูปที่ \ref{fig:package-display-stats-diagram}
โดยในระบบแสดงสถิติจะประกอบไปด้วย Use Case ดังนี้
\begin{enumerate}
    \item Use case ดูข้อมูลสถิติการทดสอบ ดังแสดงในตารางที่ \ref{tab:usecase-read-stats}
    \item Use case ส่งออกข้อมูลสถิติการทดสอบ ดังแสดงในตารางที่ \ref{tab:usecase-export-stats}
    \item Use case แก้ไขข้อมูลสถิติการทดสอบ ดังแสดงในตารางที่ \ref{tab:usecase-edit-stats}
    \item Use case ส่งออกชื่อผู้ใช้และรหัสผ่านสำหรับ Moodle ดังแสดงในตารางที่ \ref{tab:usecase-export-moodle-student}
    \item Use case ดูคะแนนการทดสอบ ดังแสดงในตารางที่ \ref{tab:usecase-read-result}
    \item Use case เพิ่มข้อมูลสถิติการทดสอบ ดังแสดงในตารางที่ \ref{tab:usecase-import-stats}
\end{enumerate}

\begin{figure}[H]
    \centering
    \includegraphics[width=100mm,scale=1.0]{diagrams/package-display-stats.png}
    \caption{แผนภาพแสดงระบบแสดงสถิติ}
    \label{fig:package-display-stats-diagram}
\end{figure}

\subsection{ระบบจัดการรายชื่อผู้เข้าแข่งขัน}

ผู้จัดทำได้สกัดความต้องการของระบบจัดการรายชื่อผู้เข้าแข่งขันออกมาเป็น Use Case ทั้งหมด 11 อัน โดยจะแบ่งเป็น 2 บทบาทการใช้งาน คือ ผู้ประสานงานและผู้ดูแลระบบ ดังแสดงในรูปที่ \ref{fig:package-manage-students-diagram}
โดยในระบบจัดการรายชื่อผู้เข้าแข่งขันจะประกอบไปด้วย Use Case ดังนี้
\begin{enumerate}
    \item Use case ค้นหารายชื่อนักเรียน ดังแสดงในตารางที่ \ref{tab:usecase-search-student}
    \item Use case อ่านรายชื่อนักเรียน ดังแสดงในตารางที่ \ref{tab:usecase-read-student}
    \item Use case ลบรายชื่อนักเรียน ดังแสดงในตารางที่ \ref{tab:usecase-remove-student}
    \item Use case แก้ไขข้อมูลนักเรียน ดังแสดงในตารางที่ \ref{tab:usecase-edit-student}
    \item Use case เพิ่มรายชื่อนักเรียน ดังแสดงในตารางที่ \ref{tab:usecase-create-student}
    \item Use case สร้างชื่อผู้ใช้และรหัสผ่านสำหรับ Moodle ดังแสดงในตารางที่ \ref{tab:usecase-generate-user}
    \item Use case เพิ่มรายชื่อนักเรียนแบบกลุ่ม ดังแสดงในตารางที่ \ref{tab:usecase-create-batch-student}
    \item Use case อ่านข้อมูลประเภทการทดสอบ ดังแสดงในตารางที่ \ref{tab:usecase-read-contest}
    \item Use case เพิ่มประเภทการทดสอบ ดังแสดงในตารางที่ \ref{tab:usecase-create-contest}
    \item Use case ส่งออกรายชื่อนักเรียน ดังแสดงในตารางที่ \ref{tab:usecase-export-student}
    \item Use case ลบประเภทการทดสอบ ดังแสดงในตารางที่ \ref{tab:usecase-remove-contest}
\end{enumerate}

\begin{figure}[H]
    \centering
    \includegraphics[width=100mm,scale=1.0]{diagrams/package-manage-students.png}
    \caption{แผนภาพแสดงระบบจัดการรายชื่อผู้เข้าแข่งขัน}
    \label{fig:package-manage-students-diagram}
\end{figure}

\subsection{Use Case Specification}

\begin{table}[H]
    \caption{Use Case อ่านสถิติการทดสอบ}
    \label{tab:usecase-read-stats}
    \begin{tabularx}{\textwidth}{ | p{3cm} | X | }
    \hline
    \textbf{Use Case} & อ่านสถิติการทดสอบ \\
    \hline
    \textbf{Actor} & ผู้ประสานงาน, ผู้ดูแลระบบ \\
    \hline
    \textbf{Description} & ผู้ใช้สามารถอ่านสถิติการทดสอบได้ \\
    \hline
    \textbf{Pre-Condition} & - \\
    \hline
    \textbf{Post-Condition} & - \\
    \hline
    \end{tabularx}
    \begin{tabularx}{\textwidth}{ | X | X | X | }
    \multicolumn{3}{|c|}{\textbf{Flow of Events}} \\
    \hline
    \multicolumn{1}{|c|}{\textbf{General}} & \multicolumn{1}{|c|}{\textbf{Actor Action}} & \multicolumn{1}{|c|}{\textbf{System Response}} \\
    \hline
    1. ผู้ใช้คลิกที่การ์ดประเภทการแข่งขันเพื่อเข้าสู่ Use Case นี้ &  &  \\
    \hline
    & 2. ผู้ใช้เลือกประเภทการแข่งขันที่ต้องการดูสถิติ & \\
    \hline
    & 3. ผู้ใช้คลิกดูสถิติ & \\
    \hline
    & & 4. ระบบสร้างการร้องขอข้อมูลสถิติแข่งขันจากฐานข้อมูลสัมพันธ์และสร้างกราฟข้อมูลให้ผู้ใช้ \\
    \hline
    \end{tabularx}
\end{table}

\begin{table}[H]
    \caption{Use Case ส่งออกข้อมูลสถิติการทดสอบ}
    \label{tab:usecase-export-stats}
    \begin{tabularx}{\textwidth}{ | p{3cm} | X | }
    \hline
    \textbf{Use Case} & ส่งออกข้อมูลสถิติการทดสอบ \\
    \hline
    \textbf{Actor} & ผู้ประสานงาน, ผู้ดูแลระบบ \\
    \hline
    \textbf{Description} & ผู้ใช้สามารถส่งออกข้อมูลสถิติการทดสอบได้ \\
    \hline
    \textbf{Pre-Condition} & - \\
    \hline
    \textbf{Post-Condition} & - \\
    \hline
    \end{tabularx}
    \begin{tabularx}{\textwidth}{ | X | X | X | }
    \multicolumn{3}{|c|}{\textbf{Flow of Events}} \\
    \hline
    \multicolumn{1}{|c|}{\textbf{General}} & \multicolumn{1}{|c|}{\textbf{Actor Action}} & \multicolumn{1}{|c|}{\textbf{System Response}} \\
    \hline
    1. ผู้ใช้คลิกที่การ์ดประเภทการแข่งขันเพื่อเข้าสู่ Use Case นี้ &  &  \\
    \hline
    & 2. ผู้ใช้เลือกประเภทการแข่งขันที่ต้องการส่งออก & \\
    \hline
    & 3. ผู้ใช้คลิกดูสถิติ & \\
    \hline
    & 4. ผู้ใช้คลิกส่งออกข้อมูลสถิติ & \\
    \hline
    & & 5. ระบบสร้างการร้องขอข้อมูลสถิติแข่งขันจากฐานข้อมูลสัมพันธ์และดาวน์โหลดให้ผู้ใช้ \\
    \hline
    \end{tabularx}
\end{table}

\begin{table}[H]
    \caption{Use Case แก้ไขข้อมูลสถิติการทดสอบ}
    \label{tab:usecase-edit-stats}
    \begin{tabularx}{\textwidth}{ | p{3cm} | X | }
    \hline
    \textbf{Use Case} & แก้ไขข้อมูลสถิติการทดสอบ \\
    \hline
    \textbf{Actor} & ผู้ดูแลระบบ \\
    \hline
    \textbf{Description} & ผู้ดูแลระบบสามารถแก้ไขข้อมูลสถิติการทดสอบได้ \\
    \hline
    \textbf{Pre-Condition} & - \\
    \hline
    \textbf{Post-Condition} & - \\
    \hline
    \end{tabularx}
    \begin{tabularx}{\textwidth}{ | X | X | X | }
    \multicolumn{3}{|c|}{\textbf{Flow of Events}} \\
    \hline
    \multicolumn{1}{|c|}{\textbf{General}} & \multicolumn{1}{|c|}{\textbf{Actor Action}} & \multicolumn{1}{|c|}{\textbf{System Response}} \\
    \hline
    1. ผู้ดูแลระบบคลิกที่การ์ดประเภทการแข่งขันเพื่อเข้าสู่ Use Case นี้ &  &  \\
    \hline
    & 2. ผู้ดูแลระบบคลิกปุ่มเพิ่มผลลัพธ์การแข่งขันเพื่อเขียนทับสถิติเดิม &  \\
    \hline
    & 3. ผู้ดูแลเพิ่มไฟล์ผลลัพธ์การแข่งขันในรูปแบบ CSV  &  \\
    \hline
    & 4. ผู้ดูแลระบบกดปุ่มเพิ่ม &  \\
    \hline
    & & 5. ระบบแก้ไขข้อมูลสถิติการทดสอบลงในฐานข้อมูลสัมพันธ์ \\
    \hline
    & & 6. ระบบแจ้งเตือนการเพิ่มสถิติการทดสอบสำเร็จแก่ผู้ดูแลระบบ \\
    \hline
    \end{tabularx}
\end{table}

\begin{table}[H]
    \caption{Use Case ส่งออกชื่อผู้ใช้และรหัสผ่านสำหรับ Moodle}
    \label{tab:usecase-export-moodle-student}
    \begin{tabularx}{\textwidth}{ | p{3cm} | X | }
    \hline
    \textbf{Use Case} & ส่งออกชื่อผู้ใช้และรหัสผ่านสำหรับ Moodle \\
    \hline
    \textbf{Actor} & ผู้ประสานงาน, ผู้ดูแลระบบ \\
    \hline
    \textbf{Description} & ผู้ใช้สามารถส่งออกชื่อผู้ใช้และรหัสผ่านสำหรับ Moodle ได้ \\
    \hline
    \textbf{Pre-Condition} & - \\
    \hline
    \textbf{Post-Condition} & - \\
    \hline
    \end{tabularx}
    \begin{tabularx}{\textwidth}{ | X | X | X | }
    \multicolumn{3}{|c|}{\textbf{Flow of Events}} \\
    \hline
    \multicolumn{1}{|c|}{\textbf{General}} & \multicolumn{1}{|c|}{\textbf{Actor Action}} & \multicolumn{1}{|c|}{\textbf{System Response}} \\
    \hline
    1. ผู้ดูแลระบบคลิกที่การ์ดประเภทการแข่งขันเพื่อเข้าสู่ Use Case นี้ &  &  \\
    \hline
    & 2. ผู้ใช้เลือกประเภทการแข่งขันที่ต้องส่งออก  &  \\
    \hline
    & 3. ผู้ใช้คลิกเลือกรูปแบบไฟล์ที่ต้องการ  &  \\
    \hline
    & & 4. ระบบเลือกชื่อผู้ใช้และรหัสผ่านของนักเรียนของโรงเรียนผู้ประสานงานจากฐานข้อมูลสัมพันธ์ \\
    \hline
    & & 5. ระบบแปลงไฟล์เป็นรูปแบบเดียวกับที่ผู้ใช้เลือก \\
    \hline
    & & 6. ระบบดาวน์โหลดให้ผู้ใช้ \\
    \hline
    \end{tabularx}
\end{table}

\begin{table}[H]
    \caption{Use Case ดูคะแนนการทดสอบ}
    \label{tab:usecase-read-result}
    \begin{tabularx}{\textwidth}{ | p{3cm} | X | }
    \hline
    \textbf{Use Case} & ดูคะแนนการทดสอบ \\
    \hline
    \textbf{Actor} & ผู้ประสานงาน, ผู้ดูแลระบบ \\
    \hline
    \textbf{Description} & ผู้ใช้สามารถดูคะแนนการทดสอบได้ \\
    \hline
    \textbf{Pre-Condition} & - \\
    \hline
    \textbf{Post-Condition} & - \\
    \hline
    \end{tabularx}
    \begin{tabularx}{\textwidth}{ | X | X | X | }
    \multicolumn{3}{|c|}{\textbf{Flow of Events}} \\
    \hline
    \multicolumn{1}{|c|}{\textbf{General}} & \multicolumn{1}{|c|}{\textbf{Actor Action}} & \multicolumn{1}{|c|}{\textbf{System Response}} \\
    \hline
    1. ผู้ใช้คลิกที่การ์ดประเภทการแข่งขันเพื่อเข้าสู่ Use Case นี้ &  &  \\
    \hline
    & 2. ผู้ใช้เลือกประเภทการแข่งขันที่ต้องการดูคะแนนการทดสอบ & \\
    \hline
    & & 4. ระบบสร้างการร้องขอข้อมูลคะแนนจากฐานข้อมูลสัมพันธ์และสร้างตารางข้อมูลให้ผู้ใช้ \\
    \hline
    \end{tabularx}
\end{table}

\begin{table}[H]
    \caption{Use Case เพิ่มข้อมูลสถิติการทดสอบ}
    \label{tab:usecase-import-stats}
    \begin{tabularx}{\textwidth}{ | p{3cm} | X | }
    \hline
    \textbf{Use Case} & เพิ่มข้อมูลสถิติการทดสอบ \\
    \hline
    \textbf{Actor} & ผู้ดูแลระบบ \\
    \hline
    \textbf{Description} & ผู้ดูแลระบบสามารถเพิ่มข้อมูลสถิติการทดสอบได้ \\
    \hline
    \textbf{Pre-Condition} & - \\
    \hline
    \textbf{Post-Condition} & - \\
    \hline
    \end{tabularx}
    \begin{tabularx}{\textwidth}{ | X | X | X | }
    \multicolumn{3}{|c|}{\textbf{Flow of Events}} \\
    \hline
    \multicolumn{1}{|c|}{\textbf{General}} & \multicolumn{1}{|c|}{\textbf{Actor Action}} & \multicolumn{1}{|c|}{\textbf{System Response}} \\
    \hline
    1. ผู้ดูแลระบบคลิกที่การ์ดประเภทการแข่งขันเพื่อเข้าสู่ Use Case นี้ &  &  \\
    \hline
    & 2. ผู้ดูแลระบบคลิกปุ่มเพิ่มผลลัพธ์การแข่งขัน  &  \\
    \hline
    & 3. ผู้ดูแลเพิ่มไฟล์ผลลัพธ์การแข่งขันในรูปแบบ CSV  &  \\
    \hline
    & 4. ผู้ดูแลระบบกดปุ่มเพิ่ม &  \\
    \hline
    & & 5. ระบบเพิ่มข้อมูลสถิติการทดสอบลงในฐานข้อมูลสัมพันธ์ \\
    \hline
    & & 6. ระบบแจ้งเตือนการเพิ่มสถิติการทดสอบสำเร็จแก่ผู้ดูแลระบบ \\
    \hline
    \end{tabularx}
\end{table}

\begin{table}[H]
    \caption{Use Case ค้นหารายชื่อนักเรียน}
    \label{tab:use-case-search-student}
    \begin{tabularx}{\textwidth}{ | p{3cm} | X | }
    \hline
    \textbf{Use Case} & ค้นหารายชื่อนักเรียน \\
    \hline
    \textbf{Actor} & ผู้ประสานงาน, ผู้ดูแลระบบ \\
    \hline
    \textbf{Description} & ผู้ใช้สามารถค้นหารายชื่อนักเรียนได้ \\
    \hline
    \textbf{Pre-Condition} & - \\
    \hline
    \textbf{Post-Condition} & - \\
    \hline
    \end{tabularx}
    \begin{tabularx}{\textwidth}{ | X | X | X | }
    \multicolumn{3}{|c|}{\textbf{Flow of Events}} \\
    \hline
    \multicolumn{1}{|c|}{\textbf{General}} & \multicolumn{1}{|c|}{\textbf{Actor Action}} & \multicolumn{1}{|c|}{\textbf{System Response}} \\
    \hline
    1. ผู้ใช้คลิกที่การ์ดนักเรียนเพื่อเข้าสู่ Use Case นี้ &  &  \\
    \hline
    & 2. ผู้ใช้กรอกข้อมูลที่ต้องการค้นหา & \\
    \hline
    & 3. ผู้ใช้กดปุ่มค้นหา &  \\
    \hline
    & & 4. ระบบแสดงรายชื่อนักเรียนที่ตรงกับเงื่อนไขการค้นหา \\
    \hline
    \end{tabularx}
\end{table}

\begin{table}[H]
    \caption{Use Case อ่านรายชื่อนักเรียน}
    \label{tab:use-case-read-student}
    \begin{tabularx}{\textwidth}{ | p{3cm} | X | }
    \hline
    \textbf{Use Case} & อ่านรายชื่อนักเรียน \\
    \hline
    \textbf{Actor} & ผู้ประสานงาน, ผู้ดูแลระบบ \\
    \hline
    \textbf{Description} & ผู้ใช้สามารถอ่านรายชื่อนักเรียนได้ \\
    \hline
    \textbf{Pre-Condition} & - \\
    \hline
    \textbf{Post-Condition} & - \\
    \hline
    \end{tabularx}
    \begin{tabularx}{\textwidth}{ | X | X | X | }
    \multicolumn{3}{|c|}{\textbf{Flow of Events}} \\
    \hline
    \multicolumn{1}{|c|}{\textbf{General}} & \multicolumn{1}{|c|}{\textbf{Actor Action}} & \multicolumn{1}{|c|}{\textbf{System Response}} \\
    \hline
    1. ผู้ใช้คลิกที่การ์ดนักเรียนเพื่อเข้าสู่ Use Case นี้ &  &  \\
    \hline
    & & 2. ระบบสร้างการร้องขอข้อมูลรายชื่อนักเรียนจากฐานข้อมูลสัมพันธ์และสร้างเป็นตารางรายชื่อให้ผู้ใช้ \\
    \hline
    \end{tabularx}
\end{table}

\begin{table}[H]
    \caption{Use Case ลบรายชื่อนักเรียน}
    \label{tab:usecase-remove-student}
    \begin{tabularx}{\textwidth}{ | p{3cm} | X | }
    \hline
    \textbf{Use Case} & ลบรายชื่อนักเรียน \\
    \hline
    \textbf{Actor} & ผู้ประสานงาน \\
    \hline
    \textbf{Description} & ผู้ใช้สามารถลบรายชื่อนักเรียนได้ \\
    \hline
    \textbf{Pre-Condition} & - \\
    \hline
    \textbf{Post-Condition} & - \\
    \hline
    \end{tabularx}
    \begin{tabularx}{\textwidth}{ | X | X | X | }
    \multicolumn{3}{|c|}{\textbf{Flow of Events}} \\
    \hline
    \multicolumn{1}{|c|}{\textbf{General}} & \multicolumn{1}{|c|}{\textbf{Actor Action}} & \multicolumn{1}{|c|}{\textbf{System Response}} \\
    \hline
    1. ผู้ใช้คลิกที่การ์ดนักเรียนเพื่อเข้าสู่ Use Case นี้ & & \\
    \hline
    & 2. ผู้ใช้คลิกที่ปุ่มถังขยะในแถวของนักเรียนที่ต้องการลบ & \\
    \hline
    & 3. ผู้ใช้คลิกที่ปุ่มยืนยัน & \\
    \hline
    & & 4. ระบบทำการลบนักเรียนจากฐานข้อมูลสัมพันธ์ \\
    \hline
    & & 5. ระบบแจ้งเตือนการลบนักเรียนสำเร็จให้แก่ผู้ใช้ \\
    \hline
    \end{tabularx}
\end{table}

\begin{table}[H]
    \caption{Use Case แก้ไขข้อมูลนักเรียน}
    \label{tab:usecase-edit-student}
    \begin{tabularx}{\textwidth}{ | p{3cm} | X | }
    \hline
    \textbf{Use Case} & แก้ไขข้อมูลนักเรียน \\
    \hline
    \textbf{Actor} & ผู้ประสานงาน \\
    \hline
    \textbf{Description} & ผู้ใช้สามารถแก้ไขข้อมูลนักเรียนได้ \\
    \hline
    \textbf{Pre-Condition} & - \\
    \hline
    \textbf{Post-Condition} & - \\
    \hline
    \end{tabularx}
    \begin{tabularx}{\textwidth}{ | X | X | X | }
    \multicolumn{3}{|c|}{\textbf{Flow of Events}} \\
    \hline
    \multicolumn{1}{|c|}{\textbf{General}} & \multicolumn{1}{|c|}{\textbf{Actor Action}} & \multicolumn{1}{|c|}{\textbf{System Response}} \\
    \hline
    1. ผู้ใช้คลิกที่การ์ดนักเรียนเพื่อเข้าสู่ Use Case นี้ & & \\
    \hline
    & 2. ผู้ใช้คลิกที่ปุ่มแก้ไขของนักเรียนที่ต้องการ & \\
    \hline
    & 3. ผู้ใช้กรอกข้อมูลของที่ต้องการแก้ไขลงในกล่องข้อความแต่ละกล่อง & \\
    \hline
    & 4. ผู้ใช้คลิกที่ปุ่มแก้ไข & \\
    \hline
    & & 5. ระบบแก้ไขข้อมูลนักเรียนลงในฐานข้อมูลสัมพันธ์ \\
    \hline
    & & 6. ระบบแจ้งเตือนการแก้ไขข้อมูลนักเรียนสำเร็จให้แก่ผู้ใช้ \\
    \hline
    \end{tabularx}
\end{table}

\begin{table}[H]
    \caption{Use Case เพิ่มรายชื่อนักเรียน}
    \label{tab:usecase-create-student}
    \begin{tabularx}{\textwidth}{ | p{3cm} | X | }
    \hline
    \textbf{Use Case} & เพิ่มรายชื่อนักเรียน \\
    \hline
    \textbf{Actor} & ผู้ประสานงาน, ผู้ดูแลระบบ \\
    \hline
    \textbf{Description} & ผู้ใช้สามารถเพิ่มรายชื่อนักเรียนได้ \\
    \hline
    \textbf{Pre-Condition} & - \\
    \hline
    \textbf{Post-Condition} & - \\
    \hline
    \end{tabularx}
    \begin{tabularx}{\textwidth}{ | X | X | X | }
    \multicolumn{3}{|c|}{\textbf{Flow of Events}} \\
    \hline
    \multicolumn{1}{|c|}{\textbf{General}} & \multicolumn{1}{|c|}{\textbf{Actor Action}} & \multicolumn{1}{|c|}{\textbf{System Response}} \\
    \hline
    1. ผู้ใช้คลิกที่การ์ดนักเรียนเพื่อเข้าสู่ Use Case นี้ & & \\
    \hline
    & 2. ผู้ใช้คลิกที่ปุ่มเพิ่มนักเรียน & \\
    \hline
    & 3. ผู้ใช้กรอกข้อมูลของนักเรียนในกล่องข้อความแต่ละกล่อง & \\
    \hline
    & 4. ผู้ใช้คลิกที่ปุ่มเพิ่ม & \\
    \hline
    & & 5. ระบบเพิ่มนักเรียนลงในฐานข้อมูลสัมพันธ์ \\
    \hline
    & & 6. ระบบแจ้งเตือนการเพิ่มนักเรียนสำเร็จให้แก่ผู้ใช้ \\
    \hline
    \end{tabularx}
\end{table}

\begin{table}[H]
    \caption{Use Case สร้างชื่อผู้ใช้และรหัสผ่านสำหรับ Moodle}
    \label{tab:usecase-generate-user}
    \begin{tabularx}{\textwidth}{ | p{3cm} | X | }
    \hline
    \textbf{Use Case} & สร้างชื่อผู้ใช้และรหัสผ่านสำหรับ Moodle \\
    \hline
    \textbf{Actor} & - \\
    \hline
    \textbf{Description} & ระบบสามารถสร้างชื่อผู้ใช้และรหัสผ่านสำหรับ Moodle ได้ \\
    \hline
    \textbf{Pre-Condition} & - \\
    \hline
    \textbf{Post-Condition} & - \\
    \hline
    \end{tabularx}
    \begin{tabularx}{\textwidth}{ | X | X | X | }
    \multicolumn{3}{|c|}{\textbf{Flow of Events}} \\
    \hline
    \multicolumn{1}{|c|}{\textbf{General}} & \multicolumn{1}{|c|}{\textbf{Actor Action}} & \multicolumn{1}{|c|}{\textbf{System Response}} \\
    \hline
    1. Use Case นี้จะเริ่มเมื่อถูกเรียกใช้จากการสร้างนักเรียนใหม่ในระบบ & & \\
    \hline
    & & 2. ระบบทำการจองชื่อผู้ใช้และสร้างรหัสผ่านใหม่ให้กับนักเรียน \\
    \hline
    & & 3. ระบบบันทึกชื่อผู้ใช้และรหัสผ่านลงในฐานข้อมูลสัมพันธ์ \\
    \hline
    \end{tabularx}
\end{table}

\begin{table}[H]
    \caption{Use Case เพิ่มรายชื่อนักเรียนแบบกลุ่ม}
    \label{tab:use-case-create-batch-student}
    \begin{tabularx}{\textwidth}{ | p{3cm} | X | }
    \hline
    \textbf{Use Case} & เพิ่มรายชื่อนักเรียนแบบกลุ่ม \\
    \hline
    \textbf{Actor} & ผู้ประสานงาน, ผู้ดูแลระบบ \\
    \hline
    \textbf{Description} & ผู้ใช้สามารถเพิ่มรายชื่อนักเรียนแบบกลุ่มได้ \\
    \hline
    \textbf{Pre-Condition} & ผู้ใช้ต้องทำการเพิ่มข้อมูลนักเรียนลงไปในไฟล์ Excel ในเทมเพลตที่จัดเตรียมไว้ให้ \\
    \hline
    \textbf{Post-Condition} & - \\
    \hline
    \end{tabularx}
    \begin{tabularx}{\textwidth}{ | X | X | X | }
    \multicolumn{3}{|c|}{\textbf{Flow of Events}} \\
    \hline
    \multicolumn{1}{|c|}{\textbf{General}} & \multicolumn{1}{|c|}{\textbf{Actor Action}} & \multicolumn{1}{|c|}{\textbf{System Response}} \\
    \hline
    1. ผู้ใช้คลิกที่การ์ดนักเรียนเพื่อเข้าสู่ Use Case นี้ & & \\
    \hline
    & 2. ผู้ใช้คลิกที่ปุ่มนำเข้านักเรียน & \\
    \hline
    & 3. ผู้ใช้คลิกที่ปุ่มอัพโหลดไฟล์ และเลือกไฟล์ Excel ที่ได้จัดเตรียมไว้ & \\
    \hline
    & 4. ผู้ใช้เลือกภาษาการสอบของนักเรียนที่ต้องการเพิ่ม & \\
    \hline
    & 5. ผู้ใช้คลิกที่ปุ่มนำเข้า & \\
    \hline
    & & 6. ระบบเพิ่มนักเรียนลงในฐานข้อมูลสัมพันธ์ \\
    \hline
    & & 7. ระบบแจ้งเตือนการเพิ่มนักเรียนสำเร็จให้แก่ผู้ใช้ \\
    \hline
    \end{tabularx}
\end{table}

\begin{table}[H]
    \caption{Use Case อ่านประเภทการแข่งขัน}
    \label{tab:use-case-read-contest}
    \begin{tabularx}{\textwidth}{ | p{3cm} | X | }
    \hline
    \textbf{Use Case} & อ่านประเภทการแข่งขัน \\
    \hline
    \textbf{Actor} & ผู้ประสานงาน, ผู้ดูแลระบบ \\
    \hline
    \textbf{Description} & ผู้ใช้สามารถอ่านประเภทการแข่งขันได้ \\
    \hline
    \textbf{Pre-Condition} & - \\
    \hline
    \textbf{Post-Condition} & - \\
    \hline
    \end{tabularx}
    \begin{tabularx}{\textwidth}{ | X | X | X | }
    \multicolumn{3}{|c|}{\textbf{Flow of Events}} \\
    \hline
    \multicolumn{1}{|c|}{\textbf{General}} & \multicolumn{1}{|c|}{\textbf{Actor Action}} & \multicolumn{1}{|c|}{\textbf{System Response}} \\
    \hline
    1. ผู้ใช้คลิกที่การ์ดประเภทการแข่งขันเพื่อเข้าสู่ Use Case นี้ &  &  \\
    \hline
    & & 2. ระบบสร้างการร้องขอข้อมูลประเภทการแข่งขันจากฐานข้อมูลสัมพันธ์และสร้างเป็นตารางประเภทการแข่งขันให้ผู้ใช้ \\
    \hline
    \end{tabularx}
\end{table}

\begin{table}[H]
    \caption{Use Case สร้างประเภทการแข่งขัน}
    \label{tab:use-case-create-contest}
    \begin{tabularx}{\textwidth}{ | p{3cm} | X | }
    \hline
    \textbf{Use Case} & สร้างประเภทการแข่งขัน \\
    \hline
    \textbf{Actor} & ผู้ดูแลระบบ \\
    \hline
    \textbf{Description} & ผู้ดูแลระบบสามารถสร้างประเภทการแข่งขันได้ \\
    \hline
    \textbf{Pre-Condition} & - \\
    \hline
    \textbf{Post-Condition} & - \\
    \hline
    \end{tabularx}
    \begin{tabularx}{\textwidth}{ | X | X | X | }
    \multicolumn{3}{|c|}{\textbf{Flow of Events}} \\
    \hline
    \multicolumn{1}{|c|}{\textbf{General}} & \multicolumn{1}{|c|}{\textbf{Actor Action}} & \multicolumn{1}{|c|}{\textbf{System Response}} \\
    \hline
    1. ผู้ดูแลระบบคลิกที่การ์ดประเภทการแข่งขันเพื่อเข้าสู่ Use Case นี้ &  &  \\
    \hline
    & 2. ผู้ดูแลระบบคลิกปุ่มสร้างประเภทการแข่งขัน  &  \\
    \hline
    & 3. ผู้ดูแลระบบกรอกข้อมูลประเภทการแข่งขัน  &  \\
    \hline
    & 4. ผู้ดูแลระบบกดปุ่มสร้าง &  \\
    \hline
    & & 5. ระบบเพิ่มข้อมูลประเภทการแข่งขันลงในฐานข้อมูลสัมพันธ์ \\
    \hline
    & & 6. ระบบแจ้งเตือนการเพิ่มประเภทการแข่งขันสำเร็จแก่ผู้ดูแลระบบ \\
    \hline
    \end{tabularx}
\end{table}

\begin{table}[H]
    \caption{Use Case ส่งออกรายชื่อนักเรียน}
    \label{tab:usecase-export-student}
    \begin{tabularx}{\textwidth}{ | p{3cm} | X | }
    \hline
    \textbf{Use Case} & ส่งออกรายชื่อนักเรียน \\
    \hline
    \textbf{Actor} & ผู้ประสานงาน, ผู้ดูแลระบบ \\
    \hline
    \textbf{Description} & ผู้ใช้สามารถส่งออกรายชื่อนักเรียนได้ \\
    \hline
    \textbf{Pre-Condition} & - \\
    \hline
    \textbf{Post-Condition} & - \\
    \hline
    \end{tabularx}
    \begin{tabularx}{\textwidth}{ | X | X | X | }
    \multicolumn{3}{|c|}{\textbf{Flow of Events}} \\
    \hline
    \multicolumn{1}{|c|}{\textbf{General}} & \multicolumn{1}{|c|}{\textbf{Actor Action}} & \multicolumn{1}{|c|}{\textbf{System Response}} \\
    \hline
    1. ผู้ใช้คลิกที่การ์ดนักเรียนเพื่อเข้าสู่ Use Case นี้ &  &  \\
    \hline
    & 2. ผู้ใช้คลิกปุ่มดาวน์โหลดรายชื่อนักเรียน  &  \\
    \hline
    & 3. ผู้ใช้คลิกเลือกรูปแบบไฟล์ที่ต้องการ  &  \\
    \hline
    & & 4. ระบบเลือกรายชื่อนักเรียนของโรงเรียนผู้ประสานงานจากฐานข้อมูลสัมพันธ์ \\
    \hline
    & & 5. ระบบแปลงไฟล์เป็นรูปแบบเดียวกับที่ผู้ใช้เลือก \\
    \hline
    & & 6. ระบบดาวน์โหลดรายชื่อให้ผู้ใช้ \\
    \hline
    \end{tabularx}
\end{table}

\begin{table}[H]
    \caption{Use Case ลบประเภทการแข่งขัน}
    \label{tab:usecase-remove-contest}
    \begin{tabularx}{\textwidth}{ | p{3cm} | X | }
    \hline
    \textbf{Use Case} & ลบประเภทการแข่งขัน \\
    \hline
    \textbf{Actor} & ผู้ดูแลระบบ \\
    \hline
    \textbf{Description} & ผู้ดูแลระบบสามารถลบประเภทการแข่งขันได้ \\
    \hline
    \textbf{Pre-Condition} & - \\
    \hline
    \textbf{Post-Condition} & - \\
    \hline
    \end{tabularx}
    \begin{tabularx}{\textwidth}{ | X | X | X | }
    \multicolumn{3}{|c|}{\textbf{Flow of Events}} \\
    \hline
    \multicolumn{1}{|c|}{\textbf{General}} & \multicolumn{1}{|c|}{\textbf{Actor Action}} & \multicolumn{1}{|c|}{\textbf{System Response}} \\
    \hline
    1. ผู้ดูแลระบบคลิกที่การ์ดประเภทการแข่งขันเพื่อเข้าสู่ Use Case นี้ &  &  \\
    \hline
    & 2. ผู้ดูแลระบบคลิกที่ปุ่มสามจุดด้านบนขวาของการ์ดประเภทการแข่งขันที่ต้องการลบ &  \\
    \hline
    & 3. ผู้ดูแลระบบกดปุ่มลบ  &  \\
    \hline
    & 4. ผู้ดูแลระบบกดยืนยันการลบประเภทการแข่งขัน &  \\
    \hline
    & & 5. ระบบลบข้อมูลประเภทการแข่งขันออกจากฐานข้อมูลสัมพันธ์ \\
    \hline
    & & 6. ระบบแจ้งเตือนการลบประเภทการแข่งขันสำเร็จแก่ผู้ดูแลระบบ \\
    \hline
    \end{tabularx}
\end{table}


\section{โครงสร้างฐานข้อมูลในระบบ (Database Schema)}

จากความต้องการของระบบแสดงสถิติและจัดการรายชื่อของผู้สมัครแข่งขันทดสอบทักษะการคิดเชิงคำนวณ สามารถสร้างออกมาเป็นฐานข้อมูลระดับโครงสร้าง ดังแสดงในรูปที่ \ref{fig:physical-database-diagram}

\begin{figure}[H]
    \centering
    \includegraphics[width=120mm,scale=1.0]{diagrams/database.png}
    \caption{แผนภาพข้อมูลระดับโครงสร้างของฐานข้อมูลในระบบ}
    \label{fig:physical-database-diagram}
\end{figure}

\section{พจนานุกรมข้อมูล (Data Dictionary)}

\subsection{ตารางข้อมูลในระบบ}
\begin{table}[htbp]
    \caption{ตารางข้อมูลในระบบ พร้อมคำอธิบาย}
    \label{tab:database}
    \begin{tabularx}{\textwidth}{ | p{3cm} | X | }
    \hline
    \textbf{ชื่อตาราง} & \textbf{คำอธิบาย} \\
    \hline
    contests & จัดเก็บประเภทการแข่งขัน \\
    \hline
    grades & จัดเก็บระดับชั้นของนักเรียน \\
    \hline
    schools & จัดเก็บข้อมูลโรงเรียนในประเทศไทย \\
    \hline
    students & จัดเก็บข้อมูลของนักเรียนที่สมัครเข้าแข่งขัน \\
    \hline
    systems & จัดเก็บสถานะการเปิด-ปิดของระบบ \\
    \hline
    users & จัดเก็บข้อมูลส่วนตัวของผู้ใช้ \\
    \hline
    user\_coordinators & จัดเก็บข้อมูลของผู้ใช้ที่เป็นผู้ประสานงานเพื่อยืนยันตัวตน \\
    \hline
    \end{tabularx}
\end{table}
    

\subsection{พจนานุกรมข้อมูลของตาราง}
\begin{table}[H]
    \caption{พจนานุกรมข้อมูลของตาราง contests}
    \label{tab:database-contests}
    \begin{tabularx}{\textwidth}{ | p{2.25cm} | p{2.20cm} | p{2.45cm} | p{2cm} | X | }
    \hline
    \textbf{ชื่อข้อมูล} & \textbf{ประเภทข้อมูล} & \textbf{ขนาดของข้อมูล} & \textbf{ข้อจำกัด} & \textbf{คำอธิบาย} \\
    \hline
    id & VARCHAR & 40 & primary\_key & รหัสการแข่งขัน \\
    \hline
    name & VARCHAR & 100 & required & ชื่อการแข่งขัน \\
    \hline
    created\_at & TIMESTAMP & 4 & - & เวลาที่สร้างข้อมูล \\
    \hline
    updated\_at & TIMESTAMP & 4 & - & เวลาที่แก้ไขข้อมูลล่าสุด \\
    \hline
    \end{tabularx}
\end{table}

\begin{table}[H]
    \caption{พจนานุกรมข้อมูลของตาราง grades}
    \label{tab:database-grades}
    \begin{tabularx}{\textwidth}{ | p{2.25cm} | p{2.20cm} | p{2.45cm} | p{2cm} | X | }
    \hline
    \textbf{ชื่อข้อมูล} & \textbf{ประเภทข้อมูล} & \textbf{ขนาดของข้อมูล} & \textbf{ข้อจำกัด} & \textbf{คำอธิบาย} \\
    \hline
    id & INT & 4 & primary\_key & รหัสระดับชั้น \\
    \hline
    name & VARCHAR & 100 & required & ชื่อระดับชั้น \\
    \hline
    created\_at & TIMESTAMP & 4 & - & เวลาที่สร้างข้อมูล \\
    \hline
    updated\_at & TIMESTAMP & 4 & - & เวลาที่แก้ไขข้อมูลล่าสุด \\
    \hline
    \end{tabularx}
\end{table}

\begin{table}[H]
    \caption{พจนานุกรมข้อมูลของตาราง schools}
    \label{tab:database-schools}
    \begin{tabularx}{\textwidth}{ | p{2.25cm} | p{2.20cm} | p{2.45cm} | p{2cm} | X | }
    \hline
    \textbf{ชื่อข้อมูล} & \textbf{ประเภทข้อมูล} & \textbf{ขนาดของข้อมูล} & \textbf{ข้อจำกัด} & \textbf{คำอธิบาย} \\
    \hline
    id & VARCHAR & 40 & primary\_key & รหัสโรงเรียน \\
    \hline
    name & VARCHAR & 200 & required & ชื่อโรงเรียน \\
    \hline
    province & VARCHAR & 150 & - & จังหวัด \\
    \hline
    district & VARCHAR & 150 & - & เขต\/อำเภอ \\
    \hline
    sub\_district & VARCHAR & 150 & - & แขวง\/ตำบล \\
    \hline
    created\_at & TIMESTAMP & 4 & - & เวลาที่สร้างข้อมูล \\
    \hline
    updated\_at & TIMESTAMP & 4 & - & เวลาที่แก้ไขข้อมูลล่าสุด \\
    \hline
    \end{tabularx}
\end{table}

\begin{table}[H]
    \caption{พจนานุกรมข้อมูลของตาราง students}
    \label{tab:database-students}
    \begin{tabularx}{\textwidth}{ | p{2.25cm} | p{2.20cm} | p{2.45cm} | p{2cm} | X | }
    \hline
    \textbf{ชื่อข้อมูล} & \textbf{ประเภทข้อมูล} & \textbf{ขนาดของข้อมูล} & \textbf{ข้อจำกัด} & \textbf{คำอธิบาย} \\
    \hline
    id & VARCHAR & 40 & primary\_key & รหัสนักเรียน \\
    \hline
    prefix & VARCHAR & 100 & required & คำนำหน้า \\
    \hline
    first\_name\_th & VARCHAR & 100 & required & ชื่อจริงภาษาไทย \\
    \hline
    last\_name\_th & VARCHAR & 100 & required & นามสกุลภาษาไทย \\
    \hline
    first\_name\_en & VARCHAR & 100 & required & ชื่อจริงภาษาอังกฤษ \\
    \hline
    last\_name\_en & VARCHAR & 100 & required & นามสกุลภาษาอังกฤษ \\
    \hline
    grade\_id & VARCHAR & 40 & foreign\_key & รหัสระดับชั้น \\
    \hline
    school\_id & VARCHAR & 40 & foreign\_key & รหัสโรงเรียน \\
    \hline
    created\_at & TIMESTAMP & 4 & - & เวลาที่สร้างข้อมูล \\
    \hline
    updated\_at & TIMESTAMP & 4 & - & เวลาที่แก้ไขข้อมูลล่าสุด \\
    \hline
    \end{tabularx}
\end{table}

\begin{table}[H]
    \caption{พจนานุกรมข้อมูลของตาราง systems}
    \label{tab:database-systems}
    \begin{tabularx}{\textwidth}{ | p{2.25cm} | p{2.20cm} | p{2.45cm} | p{2cm} | X | }
    \hline
    \textbf{ชื่อข้อมูล} & \textbf{ประเภทข้อมูล} & \textbf{ขนาดของข้อมูล} & \textbf{ข้อจำกัด} & \textbf{คำอธิบาย} \\
    \hline
    name & VARCHAR & 100 & primary\_key & ชื่อระบบย่อย \\
    \hline
    is\_online & TINTINT & 1 & required & สถานะของระบบย่อย \\
    \hline
    created\_at & TIMESTAMP & 4 & - & เวลาที่สร้างข้อมูล \\
    \hline
    updated\_at & TIMESTAMP & 4 & - & เวลาที่แก้ไขข้อมูลล่าสุด \\
    \hline
    \end{tabularx}
\end{table}

\begin{table}[H]
    \caption{พจนานุกรมข้อมูลของตาราง users}
    \label{tab:database-users}
    \begin{tabularx}{\textwidth}{ | p{2.5cm} | p{2.20cm} | p{2.45cm} | p{2.05cm} | X | }
    \hline
    \textbf{ชื่อข้อมูล} & \textbf{ประเภทข้อมูล} & \textbf{ขนาดของข้อมูล} & \textbf{ข้อจำกัด} & \textbf{คำอธิบาย} \\
    \hline
    id & VARCHAR & 40 & primary\_key & รหัสผู้ใช้ \\
    \hline
    username & VARCHAR & 100 & required & ชื่อผู้ใช้ \\
    \hline
    prefix & VARCHAR & 100 & required & คำนำหน้า \\
    \hline
    first\_name & VARCHAR & 100 & required & ชื่อจริงภาษาไทย \\
    \hline
    last\_name & VARCHAR & 100 & required & นามสกุลภาษาไทย \\
    \hline
    phone\_number & VARCHAR & 20 & required & หมายเลขโทรศัพท์ \\
    \hline
    role & ENUM & 1 & - & ระดับของผู้ใช้ \\
    \hline
    status & ENUM & 1 & - & สถานะการยืนยัน \\
    \hline
    email & VARCHAR & 255 & required & อีเมลของผู้ใช้ \\
    \hline
    created\_at & TIMESTAMP & 4 & - & เวลาที่สร้างข้อมูล \\
    \hline
    updated\_at & TIMESTAMP & 4 & - & เวลาที่แก้ไขข้อมูลล่าสุด \\
    \hline
    \end{tabularx}
\end{table}

\begin{table}[H]
    \caption{พจนานุกรมข้อมูลของตาราง user\_coordinators}
    \label{tab:database-user-coordinators}
    \begin{tabularx}{\textwidth}{ | p{1.75cm} | p{2.20cm} | p{2.45cm} | p{2cm} | X | }
    \hline
    \textbf{ชื่อข้อมูล} & \textbf{ประเภทข้อมูล} & \textbf{ขนาดของข้อมูล} & \textbf{ข้อจำกัด} & \textbf{คำอธิบาย} \\
    \hline
    user\_id & VARCHAR & 40 & primary\_key, foreign\_key & รหัสผู้ใช้ \\
    \hline
    position & TINYTEXT & 1 - 255 & required & ตำแหน่งทางการศึกษา \\
    \hline
    identity\_url & TEXT & 2 - 64Kb & required & ลิงก์ของข้อมูลยืนยันตัวตน \\
    \hline
    school\_id & VARCHAR & 40 & foreign\_key & รหัสโรงเรียน \\
    \hline
    created\_at & TIMESTAMP & 4 & - & เวลาที่สร้างข้อมูล \\
    \hline
    updated\_at & TIMESTAMP & 4 & - & เวลาที่แก้ไขข้อมูลล่าสุด \\
    \hline
    \end{tabularx}
\end{table}

\begin{table}[H]
    \caption{พจนานุกรมข้อมูลของตาราง contest\_allow\_grades}
    \label{tab:database-contest-allow_grades}
    \begin{tabularx}{\textwidth}{ | p{2.15cm} | p{2.20cm} | p{2.45cm} | p{2.15cm} | X | }
    \hline
    \textbf{ชื่อข้อมูล} & \textbf{ประเภทข้อมูล} & \textbf{ขนาดของข้อมูล} & \textbf{ข้อจำกัด} & \textbf{คำอธิบาย} \\
    \hline
    grade\_id & INT & 4 & primary\_key, foreign\_key & รหัสระดับชั้น \\
    \hline
    contest\_id & INT & 4 & primary\_key, foreign\_key & รหัสการแข่งขัน \\
    \hline
    created\_at & TIMESTAMP & 4 & - & เวลาที่สร้างข้อมูล \\
    \hline
    updated\_at & TIMESTAMP & 4 & - & เวลาที่แก้ไขข้อมูลล่าสุด \\
    \hline
    \end{tabularx}
\end{table}
\begin{table}[H]
    \caption{พจนานุกรมข้อมูลของตาราง attempts}
    \label{tab:database-attempts}
    \begin{tabularx}{\textwidth}{ | p{2.25cm} | p{2.20cm} | p{2.45cm} | p{2cm} | X | }
    \hline
    \textbf{ชื่อข้อมูล} & \textbf{ประเภทข้อมูล} & \textbf{ขนาดของข้อมูล} & \textbf{ข้อจำกัด} & \textbf{คำอธิบาย} \\
    \hline
    id & INT & 4 & primary\_key & รหัสของคำตอบในการตอบคำถาม \\
    \hline
    participate_username & VARCHAR & 40 & foreign\_key & ชื่อผู้ใช้ของผู้เข้าร่วม \\
    \hline
    score & INT & 4 & required & คะแนนรวมของคำตอบ \\
    \hline
    start\_at & TIMESTAMP & 4 & - & เวลาที่เริ่มการแข่งขัน \\
    \hline
    finished\_at & TIMESTAMP & 4 & - & เวลาที่สิ้นสุดการแข่งขัน \\
    \hline
    created\_at & TIMESTAMP & 4 & - & เวลาที่สร้างข้อมูล \\
    \hline
    updated\_at & TIMESTAMP & 4 & - & เวลาที่แก้ไขข้อมูลล่าสุด \\
    \hline
    \end{tabularx}
\end{table}
\begin{table}[H]
    \caption{พจนานุกรมข้อมูลของตาราง attempt-detailsattempt_details}
    \label{tab:database-attempt-details}
    \begin{tabularx}{\textwidth}{ | p{2.25cm} | p{2.20cm} | p{2.45cm} | p{2cm} | X | }
    \hline
    \textbf{ชื่อข้อมูล} & \textbf{ประเภทข้อมูล} & \textbf{ขนาดของข้อมูล} & \textbf{ข้อจำกัด} & \textbf{คำอธิบาย} \\
    \hline
    id & INT & 4 & primary\_key & รหัสรายละเอียดคำตอบในการตอบคำถาม \\
    \hline
    attempt\_id & INT & 4 & foreign\_key & รหัสคำตอบ \\
    \hline
    question\_id & INT & 4 & foreign\_key & รหัสคำถาม \\
    \hline
    score & INT & 4 & required & คะแนนของคำตอบ \\
    \hline
    created\_at & TIMESTAMP & 4 & - & เวลาที่สร้างข้อมูล \\
    \hline
    updated\_at & TIMESTAMP & 4 & - & เวลาที่แก้ไขข้อมูลล่าสุด \\
    \hline
    \end{tabularx}
\end{table}
\begin{table}[H]
    \caption{พจนานุกรมข้อมูลของตาราง participates}
    \label{tab:database-participates}
    \begin{tabularx}{\textwidth}{ | p{2.25cm} | p{2.20cm} | p{2.45cm} | p{2.15cm} | X | }
    \hline
    \textbf{ชื่อข้อมูล} & \textbf{ประเภทข้อมูล} & \textbf{ขนาดของข้อมูล} & \textbf{ข้อจำกัด} & \textbf{คำอธิบาย} \\
    \hline
    student\_id & VARCHAR & 40 & primary\_key, foreign\_key & รหัสนักเรียน \\
    \hline
    contest\_id & INT & 4 & primary\_key, foreign\_key & รหัสการแข่งขัน \\
    \hline
    username & VARCHAR & 40 & required & ชื่อผู้ใช้ \\
    \hline
    password & VARCHAR & 40 & required & รหัสผ่าน \\
    \hline
    created\_at & TIMESTAMP & 4 & - & เวลาที่สร้างข้อมูล \\
    \hline
    updated\_at & TIMESTAMP & 4 & - & เวลาที่แก้ไขข้อมูลล่าสุด \\
    \hline
    \end{tabularx}
\end{table}
\begin{table}[H]
    \caption{พจนานุกรมข้อมูลของตาราง questions}
    \label{tab:database-questions}
    \begin{tabularx}{\textwidth}{ | p{2.25cm} | p{2.20cm} | p{2.45cm} | p{2cm} | X | }
    \hline
    \textbf{ชื่อข้อมูล} & \textbf{ประเภทข้อมูล} & \textbf{ขนาดของข้อมูล} & \textbf{ข้อจำกัด} & \textbf{คำอธิบาย} \\
    \hline
    id & INT & 4 & primary\_key & รหัสคำถาม \\
    \hline
    name & VARCHAR & 255 & required & ชื่อคำถาม \\
    \hline
    score & INT & 4 & required & คะแนนคำถาม \\
    \hline
    penalty & FLOAT & 8 & required & คะแนนที่ลบ \\
    \hline
    type & VARCHAR & 20 & required & ประเภทคำถาม \\
    \hline
    preview\_url & TEXT & 2 - 64Kb & required & ลิงก์เพื่ออ่านคำถาม \\
    \hline
    created\_at & TIMESTAMP & 4 & - & เวลาที่สร้างข้อมูล \\
    \hline
    updated\_at & TIMESTAMP & 4 & - & เวลาที่แก้ไขข้อมูลล่าสุด \\
    \hline
    \end{tabularx}
\end{table}


\section{โครงสร้างการนำส่งระบบ (Deployment Architecture)}

การนำส่งระบบเพื่อให้ผู้ใช้งานสามารถใช้จริงบนเว็บแอปพลิเคชันนั้น ทางผู้จัดทำได้นำส่งขึ้นบน Google Cloud Platform เพื่อให้สามารถได้เซิร์ฟเวอร์ที่เสถียร ปลอดภัย และทันสมัย โดยจะเริ่มที่
เมื่อผู้จัดทำส่งโค้ดขึ้นไปบน GitHub Repository ระบบ Continuous Integration ของ GitHub Actions จะทำการ Build Container Image และสร้างการ Release ไปบน Cloud Deploy เพื่อเก็บ Container Image ไปยัง Artifact Repository และเริ่มการ Build Service บน Google Cloud Platform ด้วย Cloud Build ในระหว่างกระบวนการนี้ Cloud Build จะเก็บ Log ไว้บน Logging Service เมื่อ Build สำเร็จ Cloud Build จะสร้าง Rolling Release เพื่อนำไปปรับใช้กับ Cloud Run เพื่อทำการสร้าง Revision ให้สามารถใช้งานต่อได้ทันทีโดยไม่ต้องปิดเซิร์ฟเวอร์ และสามารถเข้าถึง API บน Cloud Run ได้ผ่าน Cloud API Gateway
ดังแสดงในรูปที่ \ref{fig:deployment-architecture}

\begin{figure}[H]
    \centering
    \includegraphics[width=120mm,scale=1.0]{diagrams/cloud.png}
    \caption{แผนภาพโครงสร้างการนำส่งระบบบน Google Cloud Platform}
    \label{fig:deployment-architecture}
\end{figure}

\subsection{ระบบย่อยทั้งหมด (Microservices)}

ในการนำส่งระบบแสดงสถิติและจัดการรายชื่อของผู้สมัครแข่งขันทดสอบทักษะการคิดเชิงคำนวณ ได้ปรับใช้การพัฒนาแบบ Microservices เพื่อช่วยให้สามารถแยกการทำงานของ API ให้ชัดเจนมากขึ้น ดังตารางที่ \ref{tab:api-mapping}

\begin{longtable}{ | p{.10\textwidth} | p{.4\textwidth} | p{.45\textwidth} | }
\caption{ตารางการแบ่งและใช้งาน Microservice}
\label{tab:api-mapping} \\
    \hline
    \textbf{ลำดับที่} & \textbf{ชื่อ Service} & \textbf{คำอธิบาย} \\
    \hline
    1. & api-retrieve & รับข้อมูลผู้ใช้งาน \\
    \hline
    2. & api-registration & สร้างผู้ใช้งานใหม่ \\
    \hline
    3. & api-list-user & แสดงรายชื่อผู้ใช้ \\
    \hline
    4. & api-verify-user & ยืนยันผู้ใช้ \\
    \hline
    5. & api-list-school & แสดงรายชื่อโรงเรียน \\
    \hline
    6. & api-create-student & สร้างนักเรียน \\
    \hline
    7. & api-create-batch-student & สร้างนักเรียนแบบกลุ่ม \\
    \hline
    8. & api-list-student & แสดงรายชื่อนักเรียน \\
    \hline
    9. & api-remove-student & ลบรายชื่อนักเรียน \\
    \hline
    10. & api-retrieve-system & รับสถานะของระบบย่อย \\
    \hline
    11. & api-export-participate & ส่งออกรายชื่อผู้เข้าแข่งขัน \\
    \hline
    12. & api-download-certificate & ดาวน์โหลดเกียรติบัตร \\
    \hline
    13. & api-edit-student & แก้ไขรายชื่อนักเรียน \\
    \hline
    14. & api-list-attempt & แสดงรายการคะแนนรวม \\
    \hline
    15. & api-list-attempt-detail & แสดงรายการคะแนนแต่ละข้อ \\
    \hline
    16. & api-list-contest & แสดงรายชื่อประเภทการแข่งขัน \\
    \hline
    17. & api-list-question & แสดงรายการคำถาม \\
    \hline
    18. & api-list-grade & แสดงรายชื่อของระดับชั้น \\
    \hline
    19. & api-list-participate & แสดงรายชื่อผู้เข้าแข่งขัน \\
    \hline
    20. & api-edit-contest & แก้ไขประเภทการแข่งขัน \\
    \hline
    21. & api-create-contest & สร้างประเภทการแข่งขัน \\
    \hline
    22. & api-create-grade & สร้างระดับชั้น \\
    \hline
    23. & api-edit-grade & แก้ไขระดับชั้น \\
    \hline
    24. & api-edit-system & แก้ไขสถานะของระบบย่อย \\
    \hline
    25. & api-create-system & สร้างสถานะของระบบย่อย \\
    \hline
    26. & api-create-school & สร้างโรงเรียน \\
    \hline
    27. & api-edit-school & แก้ไขโรงเรียน \\
    \hline
    28. & api-add-grade-allow & เพิ่มการอนุญาตระดับชั้นจากการแข่งขัน \\
    \hline
    29. & api-remove-grade-allow & ลบการอนุญาตระดับชั้นจากการแข่งขัน \\
    \hline
    30. & api-edit-user & แก้ไขข้อมูลผู้ใช้ \\
    \hline
    31. & api-create-batch-question & สร้างคำถามแบบกลุ่ม \\
    \hline
    32. & api-create-batch-attempt & สร้างคะแนนรวมแบบกลุ่ม \\
    \hline
    33. & api-create-batch-attempt-detail & สร้างคะแนนแต่ละข้อแบบกลุ่ม \\
    \hline
    34. & api-create-batch-participate & สร้างผู้เข้าแข่งขันแบบกลุ่ม \\
    \hline
    35. & api-remove-participate & ลบผู้เข้าแข่งขัน \\
    \hline
    36. & api-create-ticket & สร้างคำร้องการช่วยเหลือ \\
    \hline
    37. & api-list-ticket & แสดงรายการคำร้องการช่วยเหลือ \\
    \hline
    38. & api-retrieve-ticket & รับรายการคำร้องการช่วยเหลือ \\
    \hline
    39. & api-edit-ticket & แก้ไขรายการคำร้องการช่วยเหลือ \\
    \hline
    40. & api-create-message & ส่งข้อความในคำร้องการช่วยเหลือ \\
    \hline
    41. & api-list-message & แสดงรายการข้อความในคำร้องการช่วยเหลือ \\
    \hline
\end{longtable}

