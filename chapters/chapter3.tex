\chapter{วิธีการดำเนินการวิจัย}
\label{chapter:experiment}

\section{วิธีการดำเนินการ}

สำรวจข้อผิดพลาดของระบบเดิมที่พบเห็นอย่างละเอียด ตรวจสอบและรวบรวมความต้องการของผู้ใช้ สรุปความต้องการของผู้ใช้และออกแบบโครงสร้างและระบบของเว็บไซต์ ประชุมวางแผนงานดำเนินการตามแผนที่วางไว้ จากนั้นทำการ Deploy เพื่อให้ผู้ใช้งานได้ใช้งานจริง

\section{ความต้องการของระบบ}

\subsection{ระบบสามารถแสดงสถิติได้}

ระบบจะต้องสามารถแสดงสถิติการแข่งขันทุกครั้งที่ผ่านมาได้ โดยจะต้องสามารถดูสถิติการแข่งขันทดสอบทักษะการคิดเชิงคำนวณได้ทั้งระดับมัธยมต้นและมัธยมปลายของโรงเรียนของผู้ใช้งานได้และจะต้องสามารถแสดงสถิติการแข่งขันโดยรวมของโรงเรียนทั้งหมดได้

\subsection{ระบบจัดการรายชื่อของผู้สมัคร}

ระบบจะต้องสามารถเพิ่มข้อมูลของผู้ใช้ลงในฐานข้อมูลได้ ระบบจะต้องสามารถลบข้อมูลของผู้ใช้ออกจากฐานข้อมูลได้ ระบบจะต้องสามารถแก้ไขข้อมูลของผู้ใช้ได้

\section{คำอธิบายของระบบ}

ระบบแสดงสถิติและจัดการรายชื่อของผู้สมัครแข่งขันทดสอบทักษะการคิดเชิงคำนวณแบ่งออกเป็น 2 ส่วน ได้แก่

\subsection{ส่วนแสดงสถิติการแข่งขัน}

โดยผู้ใช้สามารถเลือกที่การ์ดการแข่งขัน (Contest) เพื่อให้แสดงข้อมูลการแข่งขันได้โดยเบื้องต้นระบบจะแสดงเป็นข้อมูลสถิติการแข่งขันทั้งหมดและข้อมูลสถิติการแข่งขันของโรงเรียนของผู้ใช้งาน โดยผู้ใช้สามารถเลือกที่จะค้นหา (Search) สถิติการแข่งขันของแต่ละโรงเรียนหรือแต่ละบุคคลได้เมื่อทำการค้นหาระบบจะแสดงข้อมูลสถิติตามการค้นหานั้น

\subsection{จัดการรายชื่อของผู้สมัครแข่งขัน}

ในกรณีที่ผู้ใช้งานคือผู้ดูแลระบบ (Admin) สามารถเลือกที่การ์ดครู (Teacher)
ในกรณีที่ผู้ใช้งานเป็นครู (Teacher) สามารถเลือกที่การ์ดนักเรียน (Student) เพื่อจัดการกับรายชื่อนักเรียนได้

\section{แผนภาพสถาปัตยกรรมเว็บแอปพลิเคชัน (Web Application Architecture)}

\section{โครงสร้างฐานข้อมูลในระบบ (Database Schema)}

\section{พจนานุกรมข้อมูล (Data Dictionary)}

\subsection{ตารางข้อมูลในระบบ}

\subsection{พจนานุกรมข้อมูลของตาราง}

\section{หลักการทำงานของระบบ}

\subsection{การเพิ่ม-ลบ-แก้ไข}

\subsection{การแสดงสถิติการแข่งขัน}