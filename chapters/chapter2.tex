\chapter{การทบทวนวรรณกรรมที่เกี่ยวข้อง}
\label{chapter:literature-review}

\section{ทฤษฎีพัฒนาซอฟแวร์}

\begin{enumerate}
    \item \textbf{Conventional Commits} คือการเขียนคอมมิต (Commit) ให้งาน โดยมีรูปแบบและกฏเกณฑ์ที่แน่นอน เพื่อให้สามารถเขียนคอมมิตได้ง่ายและมีความเข้าใจตรงกัน \cite{CICDConventionalCommits}
    \item \textbf{Git} คือตัวควบคุมเวอร์ชั่นของงานให้เป็นปัจจุบัน สามารถดูการทำงานย้อมกลับได้ในเวอร์ชั่นก่อนหน้าและมีสามารถติดตามการเปลี่ยนแปลงของงานได้ \cite{BasicGit}
    \item \textbf{Software Development Lifecycle} คือกระบวนการในการพัฒนาซอฟต์แวร์ที่ประหยัดเวลาและทำให้งานคุ้มค่าที่สุด สามารถลดความเสี่ยงต่าง ๆ และตอบสนองความต้องการได้อย่างดี ประกอบไปด้วย 6 ขั้นตอนคือ การวางแผน การออกแบบ การดำเนินการ การทดสอบ การติดตั้งใช้จริงและการรักษา \cite{WhatIsSDLC}
    \item \textbf{Agile} คือการทำงานที่รวดเร็วและมีประสิทธิภาพ โดยทุกคนสามารถทำงานร่วมกันได้ โดยการทำงานจะถูกแบ่งออกมาเป็น Sprint เพื่อให้ทุกคนในทีมทราบเป้าหมายของงานในปัจจุบันแล้วถ้าหากเกิดปัญหาจะได้สามารถแก้ไขได้อย่างรวดเร็ว \cite{WhatIsAgile}
    \item \textbf{Trunk-Based Development} คือการแบ่งงานต่าง ๆ ให้เป็นส่วนเล็ก ๆ แล้วค่อย รวม (Merge) เข้ากับสาขา (Branch) หลัก โดยควรรวมกับสาขาหลักอย่างน้อย 1 ครั้งต่อวันเพื่อลดความผิดพลาดของงาน \cite{TrunkBasedDevelopment}
    \item \textbf{Clean Architecture} คือ การเขียนโค้ด (Code) ขึ้นมาโดยแบ่งโค้ดออกเป็นชั้น (layar) ย่อย ๆ โดยแต่ละชั้นจะสามารถทำงานของตัวเองได้ 
    \item \textbf{CI/CD} CI/CD คือ กระบวนการทำงานที่ครอบคลุมการทำงานของทีม ซึ่ง CI (Continuous Integration) คือ การนำโค้ดของสมาชิกในทีมมาผ่านการรวมและการทดสอบ เพื่อทำให้มั่นใจว่าโค้ดไม่มีข้อผิดพลาดก่อนที่จะรวมโค้ดในสาขาหลัก (Branch Main) และ CD (Continuous Delivery หรือ Continuous Deployment ) คือการส่งมอบโค้ด (deploy) อย่างมีประสิทธิภาพโดยการนำโค้ดที่ผ่านการทดสอบแล้วขึ้นไปบนเซิร์ฟเวอร์ (server)
    \item \textbf{Microservice Architecture}Microservice Architecture คือ สถาปัตยกรรมการออกแบบบริการ (Service) โดยการนำบริการชิ้นใหญ่ ๆ มาแบ่งออกให้เล็กลง เพื่อลดข้อผิดพลาดของแต่ละบริการและช่วยให้สามารถทำงานชิ้นใหญ่ ๆ และซับซ้อนได้ง่ายและรวดเร็วขึ้น
    \item \textbf{Dependency Injection}Dependency Injection (DI) คือ เทคนิคในการเขียนโปแกรม (Program) ที่ช่วยลดการผูกมัด (Couple) และเพิ่มความยืดหยุ่นของโค้ด ไม่ผูกมัดกับคลาส (Class) มากจนเกินไป ทำให้ทดสอบและปรับปรุง ดูแลโค้ดได้ง่ายมากขึ้น
\end{enumerate}

\newpage

\section{เทคโนโลยีที่ใช้พัฒนา}

\subsection{Git}

Git คือ ระบบที่ถูกออกแบบมาเพื่อควบคุมเวอร์ชันของงาน (Version Control System) และสามารถติดตามการทำงาน ความเปลี่ยนแปลงต่าง ๆ ในงานได้ โดย Git จะช่วยให้สามารถพัฒนาซอฟต์แวร์ร่วมกับผู้อื่นในทีมการทำงานได้อย่างสะดวกและมีประสิทธิภาพมากขึ้น \cite{BasicGit}

\begin{figure}[H]
    \centering
    \caption{Git Logo}
    \label{fig:git}
\end{figure}

\subsection{GitHub}

GitHub คือ เว็บไซต์แพลตฟอร์มที่ช่วยในการติดตามการเปลี่ยนแปลงต่าง ๆ ภายในได้ คอยควบคุมเวอร์ชันการทำงานต่าง ๆ (Version Control) ติดตามประวัติการเปลี่ยนแปลงในโค้ดด้วย Git เป็นที่นิยมในหมู่นักพัฒนาซอฟต์แวร์ ทั้งยังสามารถทำให้ทำงานร่วมกับผู้อื่นได้ มีฟังก์ชันในการตรวจสอบและรีวิวโค้ด สามารถแบ่งปันงานของตัวเองได้ \cite{VCSGitHub}

\begin{figure}[H]
    \centering
    \caption{GitHub Logo}
    \label{fig:github}
\end{figure}

\subsection{VSCode (Visual Studio Code)}

VSCode หรือ Visual Studio Code คือ เครื่องมือสำหรับการพัฒนาซอฟต์แวร์เป็นโอเพนชอร์สเป็นตัวช่วยในการแก้ไขโค้ด (Code Editor) รองรับการทำงานหลายภาษาและสามารถติดตั้งปลั๊กอินเพิ่มเติมได้ \cite{KnowVSCode}

\begin{figure}[H]
    \centering
    \caption{Visual Studio Code Logo}
    \label{fig:vscode}
\end{figure}

\subsection{NeoVim}

NeoVim คือ โอเพนซอร์สมีการพัฒนาเป็นตัวเลือกที่ปรับปรุงมาจาก Vim (Vi IMproved) ใช้ในการแก้ไขโค้ดเน้นไปที่การใช้งานด้วยคีบอร์ด มีคำสั่งมากมายให้จดจำทำให้ใช้งานได้ยาก แต่กรณีที่ใช้จนคล่องจะเป็นประโยชน์มาก

\begin{figure}[H]
    \centering
    \caption{NeoVim Logo}
    \label{fig:neovim}
\end{figure}

\newpage

% \subsection{JetBrains WebStorm}

% JetBrains WebStorm คือ เครื่องมือที่ช่วยในการแก้ไขโค้ด (IDE - Integrated Development Environment) พัฒนาด้วยภาษา JavaScript รองรับการทำงานกับ ไลบรารี่ (Library) และเฟรมเวิร์ก (Framework) ที่หลากหลาย

% \begin{figure}[H]
%     \centering
%     \caption{JetBrains WebStorm Logo}
%     \label{fig:webstorm}
% \end{figure}

\subsection{React}

React หรือ React.js คือ ไลบรารี่ (Library) ที่เอาไว้ใช้ในการสร้าง User Interfaces สำหรับเว็บแอปพลิเคชัน พัฒนาโดย Facebook ใช้ JSX (JavaScript XML) ในการเขียน UI โค้ดในรูปแบบที่คล้ายกับ HTML ทำให้เข้าใจและใช้งานได้ง่าย \cite{ReadyToReactWithJSX}

\begin{figure}[H]
    \centering
    \caption{React Logo}
    \label{fig:react}
\end{figure}

\subsection{Next.js}

Next.js คือเฟรมเวิร์กการพัฒนาเว็บแอปพลิเคชัน (Web Application Framework) เหมือนกับ React แต่ได้รับการพัฒนาและปรับปรุงประสิทธิภาพในการโหลดหน้าทำให้ตัวเว็บแอปพลิเคชันที่เขียนด้วย Next.js มีความรวดเร็วกว่า \cite{WhatIsNextjs}

\begin{figure}[H]
    \centering
    \caption{Next.js Logo}
    \label{fig:nextjs}
\end{figure}

\newpage

\subsection{Golang}

Golang หรือ Go คือ ภาษาโปรแกรมมิ่ง (Programming Language) พัฒนาโดย Google ในปี 2007 โครงสร้างภาษาเข้าใจง่ายและอ่านง่ายกว่า มีประสิทธิภาพสูงทำให้สามารถทำงานได้อย่างรวดเร็ว \cite{WhatIsGolang}

\begin{figure}[H]
    \centering
    \caption{Golang Logo}
    \label{fig:golang}
\end{figure}

\subsection{GoFiber}

GoFiber คือ เฟรมเวิร์ก (Framework) สำหรับการพัฒนาแอปพลิเคชันเว็บและเว็บเซิร์ฟเวอร์ด้วยภาษา Go (Golang) มีความใช้งานง่าย และสามารถจัดการกับ Route ได้อย่างมีประสิทธิภาพ \cite{FiberAndGormRunGolangApp}

\begin{figure}[H]
    \centering
    \caption{GoFiber Logo}
    \label{fig:gofiber}
\end{figure}

\subsection{Prisma}

Prisma คือ เครื่องมือสำหรับจัดการฐานข้อมูล (Database Management Tool) ที่ช่วยในการเขียนคำสั่ง SQL ให้ง่ายขึ้น และสามารถเชื่อมต่อกับฐานข้อมูลได้หลากหลายชนิด \cite{WhatIsPrisma}

\begin{figure}[H]
    \centering
    \caption{Prisma Logo}
    \label{fig:prisma}
\end{figure}

\newpage

\subsection{MySQL}

MySQL คือ ระบบจัดการฐานข้อมูล (Database Management System) ที่เป็นโอเพนซอร์ส สามารถใช้งานได้ฟรี ง่ายต่อการใช้งาน และมีความเสถียรสูง \cite{WhatIsMySQL}

\begin{figure}[H]
    \centering
    \caption{MySQL Logo}
    \label{fig:mysql}
\end{figure}

\subsection{Docker}

Docker คือ เครื่องมือสำหรับจัดการและสร้าง Container ที่ใช้ในการพัฒนาแอปพลิเคชัน สามารถสร้างและใช้งานได้ง่าย มีความเร็วในการทำงานสูง และสามารถใช้งานได้ในทุกแพลตฟอร์ม \cite{WhatIsDocker}

\begin{figure}[H]
    \centering
    \caption{Docker Logo}
    \label{fig:docker}
\end{figure}

\subsection{Google Cloud Platform}

Google Cloud Platform คือ แพลตฟอร์มคลาวด์ (Cloud Platform) ที่ให้บริการในรูปแบบของ IaaS (Infrastructure as a Service) และ PaaS (Platform as a Service) พัฒนาโดย Google มีความเสถียรสูง และมีความปลอดภัยสูง \cite{GCPNewbie}

\begin{figure}[H]
    \centering
    \caption{Google Cloud Platform Logo}
    \label{fig:gcp}
\end{figure}