\chapter{บทสรุป}
\label{chapter:conclusion}

\section{สรุปผลการดำเนินงาน}

การดำเนินการของระบบแสดงสถิติและจัดการรายชื่อของผู้สมัครแข่งขันทดสอบทักษะการคิดเชิงคำนวณ โดยเริ่มดำเนินงานตั้งแต่ภาคการศึกษาที่ 1 ปีการศึกษาที่ 2566 โดยช่วงแรกของการดำเนินงานส่วนใหญ่จะเป็นการประชุมกับทีมและปรึกษาอาจารย์ ถึงปัญหาที่เกิดขึ้นของระบบเดิม หลังจากนั้นเริ่มทดสอบระบบเดิมเพื่อเก็บเกี่ยวความต้องการเพื่อที่จะนำมาปรับแก้ในการดำเนินงานครั้งใหม่ เมื่อทำการเก็บเกี่ยวความต้องการเรียบร้อย ขั้นตอนถัดมาคือการออกแบบระบบใหม่และประชุมทีมอีกครั้งเพื่อเลือกเทคโนโลยีที่ใช้ โดยตัวระบบแบ่งออกมาเป็น 2 ส่วนคือการแสดงสถิติการแข่งขันและส่วนการจัดการรายชื่อของผู้สมัครแข่งขัน

การแสดงสถิติการแข่งขันสามารถแสดงสถิติการแข่งขันในครั้งก่อนหน้าและครั้งปัจจุบันได้ โดยสามารถแสดงสถิติของนักเรียนและสถิติของโรงเรียนได้ 

การจัดการรายชื่อของผู้สมัครแข่งขันสามารถทำการเพิ่ม-ลบ-แก้ไขข้อมูลนักเรียนได้ โดยการเพิ่มผู้สมัครแข่งขันสามารถเพิ่มด้วยไฟล์ Excel (.xlsx) ได้

โดยการทำระบบทั้งสองที่กล่าวมาเสร็จภายในระยะเวลาที่ทางทีมตกลงกันเป็นอย่างดีและสามารถใช้งานได้จริงตามความต้องการที่ได้ระบุไว้ในแผนจนสมบูรณ์

\section{ปัญหาและอุปสรรคที่พบ}

\begin{enumerate}
    \item การจัดสรรเวลาที่ไม่ดีพอ ทำให้เกิดการทำงานล่าช้าในบางส่วนเนื่องมาจากการแบ่งงานที่จะทำโดยลืมคำนึงถึงระยะเวลาที่ต้องใช้ในการศึกษาเพิ่มเติม
    \item การออกแบบระบบที่ทำงานได้ไม่ตรงกับการเขียนโค้ดได้อย่างสมบูรณ์ ทำให้ต้องคอยปรับแก้อยู่หลายครั้ง
    \item การนำส่งระบบนั้นต้องใช้เวลาในการวางโครงสร้างมากกว่าที่คาดไว้ ทำให้เกิดการทำงานล่าช้าในบางส่วน
\end{enumerate}

\section{ข้อเสนอแนะ}

\begin{enumerate}
    \item ควรจัดสรรเวลาในการทำงานให้เหมาะสมกับงานที่ต้องทำ โดยคำนึงถึงระยะเวลาที่ต้องใช้ในการศึกษาเพิ่มเติม
    \item ควรทำการออกแบบระบบให้เหมาะสมกับการเขียนโค้ด โดยคำนึงถึงความสามารถของภาษาที่ใช้ในการเขียนโค้ด
\end{enumerate}