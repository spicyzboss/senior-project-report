\chapter{การทบทวนวรรณกรรมที่เกี่ยวข้อง}
\label{chapter:literature-review}

\section{ทฤษฎีพัฒนาซอฟแวร์}

\begin{enumerate}
    \item Conventional Commits
    \item Git
    \item Software Development Lifecycle
    \item Agile
    \item Trunk-Based Development
\end{enumerate}

\section{เทคโนโลยีที่ใช้พัฒนา}

\subsection{Git}

Git คือ ระบบที่ถูกออกแบบมาเพื่อควบคุมเวอร์ชันของงาน (Version Control System) และสามารถติดตามการทำงาน ความเปลี่ยนแปลงต่าง ๆ ในงานได้ โดย Git จะช่วยให้สามารถพัฒนาซอฟต์แวร์ร่วมกับผู้อื่นในทีมการทำงานได้อย่างสะดวกและมีประสิทธิภาพมากขึ้น

\subsection{GitHub}

GitHub คือ เว็บไซต์แพลตฟอร์มที่ช่วยในการติดตามการเปลี่ยนแปลงต่าง ๆ ภายในได้ คอยควบคุมเวอร์ชันการทำงานต่าง ๆ (Version Control) ติดตามประวัติการเปลี่ยนแปลงในโค้ดด้วย Git เป็นที่นิยมในหมู่นักพัฒนาซอฟต์แวร์ ทั้งยังสามารถทำให้ทำงานร่วมกับผู้อื่นได้ มีฟังก์ชันในการตรวจสอบและรีวิวโค้ด สามารถแบ่งปันงานของตัวเองได้

\subsection{VSCode (Visual Studio Code)}

VSCode หรือ Visual Studio Code คือ เครื่องมือสำหรับการพัฒนาซอฟต์แวร์เป็นโอเพนชอร์สเป็นตัวช่วยในการแก้ไขโค้ด (Code Editor) รองรับการทำงานหลายภาษาและสามารถติดตั้งปลั๊กอินเพิ่มเติมได้

\subsection{Neovim}

Neovim คือ โอเพนซอร์สมีการพัฒนาเป็นตัวเลือกที่ปรับปรุงมาจาก Vim (Vi IMproved) ใช้ในการแก้ไขโค้ดเน้นไปที่การใช้งานด้วยคีบอร์ด มีคำสั่งมากมายให้จดจำทำให้ใช้งานได้ยาก แต่กรณีที่ใช้จนคล่องจะเป็นประโยชน์มาก

\subsection{JetBrains WebStorm}

JetBrains WebStorm คือ เครื่องมือที่ช่วยในการแก้ไขโค้ด (IDE - Integrated Development Environment) พัฒนาด้วยภาษา JavaScript รองรับการทำงานกับ ไลบรารี่ (Library) และเฟรมเวิร์ก (Framework) ที่หลากหลาย

\subsection{React}

React หรือ React.js คือ ไลบรารี่ (Library) ที่เอาไว้ใช้ในการสร้าง User Interfaces สำหรับเว็บแอปพลิเคชัน พัฒนาโดย Facebook ใช้ JSX (JavaScript XML) ในการเขียน UI โค้ดในรูปแบบที่คล้ายกับ HTML ทำให้เข้าใจและใช้งานได้ง่าย

\subsection{Next.js}

Next.js คือ Next.js คือเฟรมเวิร์กการพัฒนาเว็บแอปพลิเคชัน (Web Application Framework) เหมือนกับ React แต่ได้รับการพัฒนาและปรับปรุงประสิทธิภาพในการโหลดหน้าทำให้ตัวเว็บแอปพลิเคชันที่เขียนด้วย Next.js มีความรวดเร็วกว่า

\subsection{Golang}

Golang หรือ Go คือ ภาษาโปรแกรมมิ่ง (Programming Language) พัฒนาโดย Google โครงสร้างภาษาเข้าใจง่ายและอ่านง่ายกว่า มีประสิทธิภาพสูงทำให้สามารถทำงานได้อย่างรวดเร็ว

\subsection{GoFiber}

คือ เฟรมเวิร์ก (Framework) สำหรับการพัฒนาแอปพลิเคชันเว็บและเว็บเซิร์ฟเวอร์ด้วยภาษา Go (Golang) มีความใช้งานง่ายจัดการกับ Route ได้อย่างมีประสิทธิภาพ